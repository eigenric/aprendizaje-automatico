\chapter{Modelos Lineales}

\section{Algoritmo de aprendizaje del Perceptrón (PLA)}

Implementar la función 
\mintinline{python}{ajusta_PLA(datos, label, max_iter, vini)} 
que calcula el  hiperplano solución a un problema de clasificación binaria
usando el algoritmo PLA. La entrada datos es una matriz donde cada item con su
etiqueta está representado por una fila de la matriz, label el vector de
etiquetas (cada etiqueta es un valor $+1$ o $-1$), \mintinline{python}{max_iter}
es el número máximo de iteraciones permitidas y vini el valor inicial del vector.
La función devuelve los coeficientes del hiperplano.

\subsection{Ejecutar el algoritmo PLA con los datos empleados en el apartado 2a del
ejercicio 1.}

Inicializar el algoritmo con: 

\begin{itemize}
\item el vector cero y, 
\item con vectores de números aleatorios en [0, 1] (10 veces).
\end{itemize}

Anotar el número medio de iteraciones necesarias en ambos para converger.

Se deben mostrar en una tabla cada uno de los pesos iniciales empleados, los
finales (obtenidos tras el proceso de entrenamiento), y el porcentaje de error
de clasificación. Valorar el resultado relacionando el punto de inicio con el
número de iteraciones.

\begin{table}[!ht]
    \centering
    \begin{tabular}{llll}
    \toprule
        Vector inicial & Iteraciones & Coeficientes w & Error de clasificación \\ \midrule
        $[0.000, 0.000, 0.000]$ & $75$ & $[661.00,23.20,32.39]$ & $0\%$ \\ \midrule
        $[0.574, 0.349, 0.057]$ & $257$ & $[1115.57,43.48,62.12]$ & $0\%$ \\
        $[0.229, 0.664, 0.497]$ & $43$ & $[464.23,15.39,23.75]$ & $0\%$ \\ 
        $[0.519, 0.175, 0.571]$ & $231$ & $[1078.52,39.47,53.76]$ & $0\%$ \\
        $[0.997, 0.817, 0.594]$ & $71$ & $[664.00,23.15,31.90]$ & $0\%$ \\ 
        $[0.976, 0.902, 0.596]$ & $76$ & $[661.98,24.90,36.20]$ & $0\%$ \\ 
        $[0.032, 0.094, 0.065]$ & $59$ & $[558.03,19.36,29.71]$ & $0\%$ \\ 
        $[0.452, 0.375, 0.975]$ & $274$ & $[1145.45,40.28,60.81]$ & $0\%$ \\
        $[0.168, 0.973, 0.767]$ & $235$ & $[1089.17,39.45,53.53]$ & $0\%$ \\
        $[0.824, 0.633, 0.669]$ & $257$ & $[1148.82,39.90,60.95]$ & $0\%$ \\
        $[0.477, 0.013, 0.353]$ & $74$ & $[673.48,22.59,31.35]$ & $0\%$ \\ \bottomrule
        Promedio & $157.7$ & ~ & $0\%$ \\ \bottomrule
    \end{tabular}
\end{table}

\subsection{Hacer lo mismo usando los datos del apartado 2b del ejercicio 1.}

\textbf{¿Observa algún comportamiento diferente? En caso afirmativo diga cuál y las
razones para que ello ocurra.}

\begin{table}[!ht]
    \centering
    \begin{tabular}{llll}
    \toprule
        Vector inicial & Iteraciones & Coeficientes w & Error de clasificación \\ \midrule
        $[0.000, 0.000, 0.000]$ & $5000$ & $[339.00,24.81,55.86]$ & $24\%$ \\ \midrule
        $[0.492, 0.730, 0.469]$ & $5000$ & $[348.49,17.77,42.43]$ & $24\%$ \\ 
        $[0.457, 0.138, 0.011]$ & $5000$ & $[349.46,16.66,44.13]$ & $24\%$ \\ 
        $[0.758, 0.320, 0.984]$ & $5000$ & $[339.76,21.61,56.29]$ & $22\%$ \\ 
        $[0.220, 0.339, 0.524]$ & $5000$ & $[336.22,17.72,43.85]$ & $25\%$ \\ 
        $[0.755, 0.464, 0.125]$ & $5000$ & $[336.75,17.32,53.11]$ & $27\%$ \\ 
        $[0.313, 0.505, 0.674]$ & $5000$ & $[339.31,15.10,26.87]$ & $20\%$ \\ 
        $[0.770, 0.130, 0.023]$ & $5000$ & $[333.77,27.72,49.69]$ & $20\%$ \\ 
        $[0.519, 0.810, 0.013]$ & $5000$ & $[343.52,12.09,38.81]$ & $22\%$ \\ 
        $[0.672, 0.687, 0.449]$ & $5000$ & $[331.67,12.87,21.73]$ & $19\%$ \\ 
        $[0.915, 0.644, 0.005]$ & $5000$ & $[341.91,21.10,56.13]$ & $22\%$ \\ \bottomrule
        Promedio & $5000$ & ~ & $22.5\%$ \\ \bottomrule
    \end{tabular}
\end{table}


\section{Regresión Logística (RL).}

En este ejercicio emplearemos nuestra propia función objetivo $f$ y un conjunto
de datos $D$ para ver cómo funciona regresión logística. Consideraremos d = 2
para que los datos sean fácilmente visualizables, y emplearemos 
$X = [0, 2] \times [0, 2]$ con probabilidad uniforme de elegir cada $x \in X$.
Elegir una línea en el plano que pase por X como la frontera que separa la
región en donde y toma valores $+1$ y $-1$.

Para ello, seleccionar dos puntos aleatorios de $X$ y calcular la línea que pasa
por ambos.

Impleméntese RL con Gradiente Descendente Estocástico (SGD) del siguiente modo:

\begin{itemize}
\item Inicializar el vector de pesos con valores 0.
\item Parar el algoritmo cuando $\Vert w(t+1) - w(t) \Vert < 0.01$, donde $w(t)$
denota el vector de pesos al final de la época $t$. Recuérdese que una época es
un pase completo a través de los N ejemplos de nuestro conjunto de datos. 
\item Aplicar una permutación aleatoria de $\{1, 2,..., N\}$ a los índices de
los datos, antes de usarlos en cada época del algoritmo. 
\end{itemize}

A continuación, empleando la implementación anterior, realícese el siguiente
experimento: 

\begin{itemize}
\item Seleccione $N = 100$ puntos aleatorios $\{x_n \}$  de $X$ y evalúe las
respuestas $\{y_n \}$ de todos ellos respecto de la frontera elegida.
\item Ejecute RL para encontrar la función solución $g$, y evalúe el error 
$E_{out}$ usando para ello una nueva muestra de datos ($> 999$). Se debe
escoger experimentalmente tanto el learning rate (tasa de aprendizaje $\eta$)
como el tamaño de batch. 
\item Repita el experimento $100$ veces, y calcule los valores promedio de 
$E_{out}$, de porcentaje de error de clasificación, y de épocas necesarias para
converger.
\end{itemize}