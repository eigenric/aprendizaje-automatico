\chapter{Ejercicio sobre la complejidad de h y el ruido}

En este ejercicio debemos aprender la dificultad que introduce la aparición de
ruido en las etiquetas a la hora de elegir la clase de funciones más adecuada.
Haremos uso de tres funciones incluidas en
el fichero template trabajo2.py:

\begin{itemize}

\item \mintinline{python}{simula_unif(N, dim, rango)}: calcula una lista de N vectores de dimensión
dim. Cada vector contiene dim números aleatorios uniformes en el intervalo
rango. 

\item \mintinline{python}{simula_gauss(N, dim, sigma)}: calcula una lista de
longitud N de vectores de dimensión dim, donde cada posición del vector contiene
un número aleatorio extraído de una distribucción Gaussiana de media $0$ y
varianza dada, para cada dimension, por la posición del vector sigma.  

\item \mintinline{python}{simula_recta(intervalo)}: que simula de forma
aleatoria los parámetros, $v = (a, b)$ de una recta, $y = ax + b$, que corta al
cuadrado $[−50, 50] \times [−50, 50]$.

\section{Ejercicio 1}
\textbf{Dibujar gráficas con las nubes de puntos simuladas con las siguientes
condiciones:}

\subsection{simula_unif}
\textbf{Considere $N = 50$, $dim = 2$, $rango = [−50, 50]$ con 
\mintinline{python}{simula_unif (N, dim, rango)}}


\subsection{simula_gauss}
\textbf{Considere $N = 50$, $dim = 2$ y $sigma = [5, 7]$ con 
\mintinline{python}{simula_gauss(N, dim, sigma)}}


\section{Ejercicio 2}

Vamos a valorar la influencia del ruido en la selección de la complejidad de la
clase de funciones.  Con ayuda de la función
\mintinline{python}{simula_unif(100, 2, [−50, 50])} generamos una muestra de
puntos 2D a los que vamos añadir una etiqueta usando el signo de la función 
$f(x, y) = y − ax − b$, es decir el signo de la distancia de cada punto a la
recta simulada con \mintinline{python}{simula_recta()}.  

\subsection{Dibujo de puntos con etiqueta y recta usada}

Dibujar un gráfico 2D donde los puntos muestren (use colores) el resultado de su
etiqueta. Dibuje también la recta usada para etiquetar. Observe que todos los
puntos están bien clasificados respecto de la recta.

\subsection{Añadir ruido aleatorio}

Modifique de forma aleatoria un $10\%$ de las etiquetas positivas y otro $10\%$
de las negativas y guarde los puntos con sus nuevas etiquetas. Dibuje de nuevo
la gráfica anterior. Ahora habrá puntos mal clasificados respecto de la recta.  


\subsection{Otras fronteras de clasificación}

Supongamos ahora que las siguientes funciones definen la frontera de
clasificación de los puntos de la muestra en lugar de una recta 

\begin{itemize}
    
\item $f(x,y) = (x − 10)^2 + (y − 20)^2 − 400$
\item $f(x,y) = 0.5(x + 10)^2 + (y − 20)^2 − 400$
\item $f(x,y) = 0.5(x − 10)^2 − (y + 20)^2 − 400$
\item $f(x,y) = y − 20x^2 − 5x + 3$

Visualizar el etiquetado generado en el apartado 2b junto con la gráfica de cada
una de las funciones. 
Comparar las regiones positivas y negativas de estas nuevas funciones con las
obtenidas en el caso de la recta. 
Argumente si estas funciones más complejas son
mejores clasificadores que la función lineal.
Observe las gráficas y diga qué consecuencias extrae sobre la influencia de la
modificación de etiquetas en el proceso de aprendizaje.  
Explique el razonamiento. 